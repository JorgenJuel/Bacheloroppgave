\begin{definisjon}
Et \textbf{projektivt plan} er alle ikke-null tripler $[x, y, z]$ i det projektive 2-rommet $\Projective^2$
\end{definisjon}

\begin{definisjon}
En \textbf{kurve} $C/k$ i det projektive planet er et homegent polynom $f(x,y,z)$ med koefisienter i $k$. Alle kropper $K$ som inneholder $k$ gir opphav til de $\textbf{K-rasjonale}$ punktene $$C(K) = \{[x, y, z] \in \Projective^2(K) | f(x,y,z) = 0\}$$
\end{definisjon}

Det er altså en algebraisk delmengde av $\Projective^n(K)$, dersom i tillegg polynomet $f(x,y,z)$ er primitivt har vi at $C(k)$ er en varietet.

\begin{definisjon}
En kurve $C/k$ er \textbf{singuær} i et punkt $P \in C(K)$ dersom $ \frac{\delta f}{\delta x} = \frac{\delta f}{\delta y} = \frac{\delta f}{\delta z} = 0$ når de evalueres i punktet $P$

Hvis kurven $C/k$ ikke har noen singulære punkter sier vi at den er \textbf{glatt} (eller ikke-singulær).
\end{definisjon}
